\newpage
\chapter{PENDAHULUAN} \label{Bab I}

\section{Latar Belakang} \label{I.Latar Belakang}
Organisasi dan institusi, termasuk perguruan tinggi, instansi pemerintahan, serta perusahaan bergantung pada dokumen operasional berformat PDF, seperti Standar Operasional Prosedur (SOP), peraturan, buku petunjuk kerja, dan buku pedoman, untuk menjamin konsistensi proses, kepatuhan regulasi, serta transfer pengetahuan. Namun, skala dan kompleksitas dokumen tersebut (banyak bab, istilah teknis, dan rujukan silang) menyulitkan pengguna ketika harus menelusuri jawaban spesifik secara cepat dan tepat; pencarian manual memakan waktu dan rawan salah tafsir karena informasi sering tersebar di beberapa bagian yang saling terkait.

Pendekatan pencarian berbasis kata kunci yang lazim digunakan pada PDF memiliki keterbatasan karena hanya mengandalkan kecocokan leksikal. Pendekatan ini tidak mampu menangkap sinonimi, parafrasa, dan hubungan semantik antarkalimat, serta kurang efektif ketika pertanyaan menuntut pemahaman konteks lintas paragraf atau subbab. Kesenjangan ini menegaskan kebutuhan akan solusi yang melampaui sekadar pencarian kata, yakni solusi yang benar-benar memahami isi dokumen dan maksud pertanyaan pengguna.

Kemajuan Pemrosesan Bahasa Alami (Natural Language Processing/NLP) berbasis arsitektur \textit{transformer} membuka peluang untuk membangun sistem \textit{question answering} (QA) yang mampu memahami pertanyaan dan mengekstraksi jawaban dari konteks dokumen panjang secara lebih presisi \cite{vaswani2017attention,devlin2019bert}. Di ranah bahasa Indonesia, keluarga model IndoBERT, termasuk varian ringan IndoBERT-Lite, menyediakan fondasi efisien untuk diadaptasi (\textit{fine-tuning}) ke tugas QA sehingga tetap terjangkau dari sisi komputasi \cite{wilie2020indobert}. Selain itu, integrasi teknik pengambilan informasi (\textit{retrieval}) sebelum inferensi, sebagaimana ditunjukkan pada paradigma \textit{retrieval-augmented}, dapat membantu sistem menjawab pertanyaan berbasis pengetahuan yang tersebar di dokumen besar \cite{lewis2020rag}, sementara studi QA atas dokumen panjang/terstruktur terus menyoroti tantangan praktis terkait segmentasi konteks, pemilihan paragraf relevan, dan keterlacakan sumber (mis. PDF/long-document QA).

Berdasarkan tinjauan tersebut, penelitian ini mengangkat dua \textit{research gap}: (1) belum ada penelitian yang secara khusus mengevaluasi IndoBERT-Lite untuk QA pada dokumen PDF nyata berbahasa Indonesia; (2) belum ada integrasi dan evaluasi sistematis pipeline \textit{retrieval} + QA yang ditujukan untuk dokumen non-dataset (PDF) institusional. Untuk menjawab gap ini, penelitian memanfaatkan model pralatih \texttt{Wikidepia/indobert-lite-squad} sebagai titik awal \textit{fine-tuning} dan dataset \texttt{jakartaresearch/indoqa} sebagai sumber pasangan konteks–pertanyaan–jawaban berbahasa Indonesia, dengan fokus evaluasi pada skenario dunia nyata.
\section{Rumusan Masalah} \label{I.Rumusan Masalah}
Berdasarkan latar belakang, penelitian ini difokuskan pada perancangan dan evaluasi sistem QA berbahasa Indonesia untuk dokumen PDF institusional dalam skala yang realistis untuk diselesaikan. Rumusan masalah yang dikaji adalah:

\begin{enumerate}[noitemsep,label=\textbf{RQ\arabic*:}]
	\item Bagaimana mengembangkan pipeline QA berbasis \textit{retrieval} yang menerima input PDF + pertanyaan, mengekstraksi dan menyegmentasi teks, melakukan \textit{indexing} serta pemilihan paragraf relevan, dan menghasilkan jawaban beserta referensi paragraf sumber sebagai keluaran yang terstandar?
	\item Bagaimana performa model IndoBERT-Lite yang di-\textit{fine-tune} pada dataset IndoQA (diukur dengan metrik EM/F1) ketika digunakan sebagai penjawab ekstraktif pada konteks paragraf hasil \textit{retrieval}, dibandingkan performa pralatihnya?
	\item Seberapa akurat sistem end-to-end dalam menjawab pertanyaan berbasis dokumen PDF nyata (mis. SOP atau pedoman) yang dipilih, diukur dengan EM/F1 per pertanyaan, serta apa sumber kesalahan utama dari tahap \textit{retrieval} maupun ekstraksi jawaban?
\end{enumerate}

Rumusan masalah ini akan dijawab melalui rancangan sistem, implementasi perangkat lunak, eksperimen terukur, dan analisis yang dilaporkan sampai \nameref{V.Kesimpulan}.

\section{Tujuan Penelitian} \label{I.Tujuan}
Merujuk pada \nameref{I.Latar Belakang} dan \nameref{I.Rumusan Masalah}, tujuan penelitian ini dirumuskan sebagai berikut:
\begin{enumerate}[noitemsep,label=\textbf{RO\arabic*:}]
	\item Mengembangkan pipeline QA berbasis \textit{retrieval} untuk dokumen PDF berbahasa Indonesia, mencakup ekstraksi/segmentasi PDF, \textit{indexing}, pemilihan paragraf relevan, serta penyajian jawaban berikut referensi paragraf sumber.
	\item Memperoleh model QA IndoBERT-Lite hasil \textit{fine-tuning} pada dataset IndoQA, dan mengukur peningkatan kinerjanya (EM/F1) dibandingkan model pralatih.
	\item Mengevaluasi performa sistem QA end-to-end pada dokumen PDF target (mis. SOP atau pedoman), diukur dengan EM/F1 per pertanyaan serta analisis kesalahan dari tahap \textit{retrieval} dan ekstraksi jawaban.
\end{enumerate}

\section{Batasan Masalah} \label{I.Batasan}
Batasan yang dimaksud disini ialah batasan dari penelitian tugas akhir yang dilakukan. Batasan masalah ditujukan agar tugas akhir yang dilakukan tidak terlalu luas, dan menjadi lebih realistis untuk diselesaikan. \par

\begin{enumerate}[noitemsep]
	\item Dokumen yang diuji adalah PDF berbasis teks (bukan hasil pemindaian tanpa OCR), sehingga ekstraksi teks tidak membahas pipeline OCR.
	\item Task QA bersifat ekstraktif (jawaban diambil langsung dari paragraf konteks) dan tidak mencakup generatif bebas.
	\item Model utama yang digunakan adalah IndoBERT-Lite; model lain hanya sebagai pembanding ringan jika diperlukan.
	\item Data \textit{fine-tuning} utama adalah dataset IndoQA; penyesuaian domain hanya sebatas format PDF target, tanpa kurasi dataset baru berskala besar.
	\item Evaluasi kinerja menggunakan metrik EM dan F1 untuk level jawaban, serta analisis kesalahan pada tahap \textit{retrieval} dan ekstraksi jawaban.
\end{enumerate}

\section{Manfaat Penelitian} \label{I.Manfaat}
Manfaat yang ingin dicapai, diharapkan dapat memberikan dampak positif terhadap mahasiswa, program studi Teknik Informatika, ITERA, pengguna institusional, maupun komunitas akademik/peneliti, ialah sebagai berikut: \par

\begin{enumerate}[noitemsep]
	\item Tersedianya pipeline QA PDF berbahasa Indonesia yang dapat diterapkan pada dokumen institusional (mis. SOP, peraturan kampus/perusahaan), sehingga akses informasi menjadi lebih cepat dan presisi.
	\item Model QA berbasis IndoBERT-Lite yang telah di-\textit{fine-tune} dan tervalidasi, memberikan referensi performa (EM/F1) untuk riset dan pengembangan lebih lanjut.
	\item Pemanfaatan dataset internal PDF untuk QA yang selama ini jarang dieksplorasi, membuka peluang studi lanjutan tentang domain adaption dan evaluasi QA pada dokumen dunia nyata.
	\item Bagi mahasiswa/program studi: pengalaman merancang, mengimplementasikan, dan mengevaluasi sistem AI terapan dengan batasan sumber daya; bagi akademisi/peneliti: kontribusi studi kasus QA PDF berbahasa Indonesia yang dapat direplikasi atau diperluas.
\end{enumerate}

\section{Sistematika Penulisan} \label{I.Sistematika}
Sistematika penulisan berisi pembahasan apa yang akan ditulis disetiap Bab. Sistematika pada umumnya berupa paragraf yang setiap paragraf mencerminkan bahasan setiap Bab. \par

\subsection*{Bab I}
Bab ini berisikan penjelasan latar belakang dari topik penelitian yang berlangsung, rumusan masalah dari masalah yang dihadapi pada penjelasan di latar belakang, tujuan dari penelitian, batasan dari penelitian, manfaat dari hasil penelitian, dan sistematika penulisan tugas akhir. \par

\subsection*{Bab II}
Bab ini membahas mengenai tinjauan pustaka dari penelitan terdahulu dan dasar teori yang berkaitan dengan penelitian ini.

\subsection*{Bab III}
Bab ini berisikan penjelasan alur kerja sistem, alat dan data yang digunakan, metode yang digunakan, dan rancangan pengujian.

\subsection*{Bab IV}
Bab ini membahas hasil implementasi dan pengujian dari penelitian yang dilakukan, serta analisis dan evaluasi yang dapat dipetik dari hasil.

\subsection*{Bab V}
Bab ini membahas kesimpulan dari hasil penelitian dan juga saran untuk penelitian selanjutnya.
