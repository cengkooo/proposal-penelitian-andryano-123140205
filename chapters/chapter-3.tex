\newpage
\chapter{METODE PENELITIAN} \label{Bab III}

\section{Alur Penelitian} \label{III.Alur}
Penelitian ini merancang dan mengevaluasi sistem QA PDF berbahasa Indonesia berbasis
\textit{transformer} dengan \textit{fine-tuning} IndoBERT-Lite pada IndoQA
\cite{wilie2020indobert,indoqa2023}. Kerangka kerja mengikuti tahapan:
\begin{enumerate}[noitemsep]
    \item Akuisisi dan kurasi data (IndoQA + PDF institusional).
    \item Ekstraksi \& segmentasi PDF (layout-aware dan \textit{fixed window}).
    \item Indeks \& \textit{retrieval} (BM25 baseline; opsional dense).
    \item \textit{Reader} ekstraktif (IndoBERT-Lite di-\textit{fine-tune}).
    \item Evaluasi berlapis (komponen, end-to-end, ablation, V\&V).
\end{enumerate}

\section{Metode Pengumpulan Data} \label{III.Pengumpulan}
Sumber data sekunder. Dataset IndoQA \cite{indoqa2023} (konteks--pertanyaan--
jawaban) digunakan untuk \textit{fine-tuning} dan validasi reader, dengan split
train/val/test (80/10/10 atau sesuai kartu dataset).

Sumber data dari file atau dokumen berupa PDF. Dipilih 1--2 PDF institusional (SOP/peraturan)
berteks (non-scan), tebal $\geq$30 halaman, terstruktur (bab/subbab). Ekstraksi teks
memakai \texttt{pdfminer.six}/\texttt{PyPDF2}; OCR dikecualikan sesuai batasan.

Gold standard PDF. Disusun 100--200 pasang Q--A dari PDF oleh dua anotator;
kesepakatan dihitung dengan Cohen’s $\kappa$ (target $\geq$ 0{,}75). Pertanyaan disusun
berdasar kebutuhan domain; data sensitif dihapus, memakai dokumen publik/izin internal
dan atribusi sumber.

\section{Metode Perancangan/Pengembangan} \label{III.Pengembangan}
\subsection{Arsitektur Sistem}
Modular: ingestion PDF $\rightarrow$ segmentasi $\rightarrow$ indeks $\rightarrow$
retrieval $\rightarrow$ reader (QA) $\rightarrow$ pelacakan sumber $\rightarrow$
antarmuka. Konsep \textit{retrieval-augmented} \cite{lewis2020rag,karpukhin2020dpr} dan
\textit{transformer} \cite{vaswani2017attention,devlin2019bert,wilie2020indobert} menjadi
dasar, sementara PDFTriage menekankan struktur dokumen \cite{saad2023pdftriage}.

\subsection{Pra-pemrosesan dan Segmentasi}
\begin{itemize}
    \item Normalisasi karakter (Unicode, spasi), penggabungan tanda baca per kalimat/
    paragraf. Untuk retrieval sparse: \textit{lowercase}, normalisasi angka/tanggal,
    \textit{stopword} opsional; untuk reader: teks dipertahankan apa adanya.
    \item Segmentasi layout-aware: deteksi heading/subheading, pecah per subbagian atau
    paragraf logis (150--250 kata). Fallback: jendela 384 token dengan \textit{stride}
    128. Struktur membantu keterlacakan sumber \cite{saad2023pdftriage}.
\end{itemize}

\subsection{Indeks dan Retrieval}
\begin{itemize}
    \item BM25 (Pyserini/Lucene) sebagai baseline: $k1\approx0{,}9$--$1{,}2$, $b\approx
    0{,}4$--$0{,}75$, \textit{top-k} awal = 10.
    \item Opsional dense retrieval (mis. bi-encoder SEA-LION) untuk studi banding
    Recall@k.
    \item \textit{Top-k} (5--10) paragraf kandidat dikirim ke reader.
\end{itemize}

\subsection{Model QA dan Fine-tuning}
\begin{itemize}
    \item Basis: \texttt{Wikidepia/indobert-lite-squad} (extractive QA).
    \item Data: IndoQA train/val/test.
    \item Hiperparameter awal: \texttt{max\_seq\_length}=384, \texttt{doc\_stride}=128,
    \texttt{max\_answer\_length}=30, \texttt{batch\_size}=16, \texttt{lr}=2e-5,
    \texttt{epochs}=3--5, \texttt{weight\_decay}=0.01, \texttt{warmup\_ratio}=0.1, fp16,
    \texttt{seed}=42, \textit{early stopping} berdasar F1 (patience 2).
\end{itemize}

\subsection{Inferensi End-to-End}
Menerima pertanyaan $\rightarrow$ \textit{retrieve} \textit{top-k} paragraf $\rightarrow$
reader memproduksi span jawaban + skor $\rightarrow$ pilih skor tertinggi $\rightarrow$
kembalikan jawaban beserta paragraf/halaman sumber.

\subsection{Kontrol Versi \& Reproduksibilitas}
\begin{itemize}
    \item Berkas lingkungan (\texttt{requirements.txt}), konfigurasi YAML untuk
    hiperparameter, \textit{random seed}, dan jejak \textit{commit} model.
    \item \textit{Experiment tracking} (MLflow/Weights \& Biases) untuk metrik.
\end{itemize}

\section{Metode Pengujian/Validasi} \label{III.Pengujian}
\subsection{Evaluasi Komponen}
\begin{itemize}
    \item Retrieval-only: Recall@k (k $\in\{1,3,5,10\}$), MRR@k terhadap paragraf emas.
    \item Reader-only (IndoQA): EM dan F1 pada val/test IndoQA.
\end{itemize}

\subsection{Evaluasi End-to-End PDF}
\begin{itemize}
    \item Dataset: himpunan Q--A anotasi PDF.
    \item Prosedur: retrieval $\rightarrow$ reader per pertanyaan; ukur EM/F1; catat
    paragraf/halaman sumber dan klasifikasi kesalahan (retrieval vs ekstraksi).
\end{itemize}

\subsection{Uji Statistik dan Ablasi}
\begin{itemize}
    \item Bootstrap 1000$\times$ untuk CI 95\% EM/F1; uji McNemar untuk dua sistem.
    \item Ablasi: segmentasi layout-aware vs paragraf polos; variasi \textit{top-k}
    (5/10); BM25 vs dense retrieval.
\end{itemize}

\subsection{Validasi Ahli \& Kegunaan (opsional)}
\begin{itemize}
    \item \textit{Expert review}: 20 sampel jawaban dinilai 3 penilai
    (benar/parsial/salah) + justifikasi.
    \item Uji kegunaan ringan (SUS) pada 8--12 pengguna internal untuk prototipe
    antarmuka tanya--jawab.
\end{itemize}

\subsection{Kriteria Penerimaan}
\begin{itemize}
    \item Reader (IndoQA): F1 $\geq$ baseline publik HF.
    \item End-to-end PDF: F1 $\geq$ 0{,}60 dan \textit{traceability} $\geq$ 95\%.
    \item Retrieval PDF: Recall@10 $\geq$ 0{,}85.
\end{itemize}

\subsection{Risiko dan Mitigasi}
\begin{itemize}
    \item PDF scan/gambar: dikecualikan; OCR hanya pekerjaan lanjutan.
    \item Jawaban sangat pendek: atur \texttt{max\_answer\_length} dan normalisasi
    pasca-proses.
    \item Batas komputasi: gunakan IndoBERT-Lite, fp16, \textit{gradient accumulation}.
\end{itemize}
